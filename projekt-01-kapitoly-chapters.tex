% Tento soubor nahraďte vlastním souborem s obsahem práce.
%=========================================================================
% Autoři: Michal Bidlo, Bohuslav Křena, Jaroslav Dytrych, Petr Veigend a Adam Herout 2019
%------------CHAPTER--------------
\chapter{Úvod}
Cílem práce je navrhnout, implementovat a otestovat aplikaci určenou pro operační systém Android. Tato aplikace bude uživateli zprostředkovávat verzovací nástroj Git formou přívětivého grafického rozhraní. Dále bude implementovat rozšíření \emph{git-lfs} a \emph{git-annex} za účelem snížení velikosti repozitářů. Aplikace je určena zejména vývojářům a pokročilým uživatelům. Je tedy navržena jako maximálně transparentní při zachování jednoduchého ovládání mobilního zařízení.

\subsection {Git}
Git slouží zejména programátorům k verzování jejich práce, popřípadě jejího sdílení s ostatními členy týmu. Nicméně jeho využití je široké a to zejména při využití rozšíření git-lfs nebo git-annex, která se zaměřují na práci s velkými soubory.

\subsection{Git LFS}
Git Large File Storage (LFS) nahrazuje velké soubory v repozitářích ukazateli. Samotné soubory jsou pak uloženy na vzdáleném serveru.

\subsection{Git Annex}
Git annex řeší stejný problém jako Git LFS, ale mírně odlišným způsobem. V repozitáři je uložen symbolický odkaz na klíč, který je hash daného souboru. Samotný soubor je pak uložen v adresáři \emph{.git/annex/}. Při změně souboru se mění jen jeho hash a aktualizuje symbolický odkaz. Tímto způsobem je zajištěno šetření místa, jelikož samotný soubor je v repozitáři uložen maximálně jeden.

%------------CHAPTER--------------
\chapter{Průzkum existujících řešení}
Během průzkumu již existující aplikací jsem se zaměřil jak na aplikace operačního systému Android, tak na desktopové operační systémy Linux a Windows.

    \subsection {Android}
    Pro operační systémy Android je trh s řešeními Gitu velice omezený. Existují zde aplikace s podporou pouze pro čtení repozitáře či takové, které zvládají i ostatní základní příkazy Gitu.
    \subsubsection{MGit}
    Za zmínku z nich stojí \emph{MGit}. Bohužel neposkytuje podporu pro \emph{git-lfs} ani \emph{git-annex}. K implementaci funkcí Gitu využívá knihovnu \emph{JGit}. Ta sice v aktuální verzi podporuje \emph{git-lfs}, ale v té, kterou aplikace využívá ji ještě nemá. Ale 
    má otevřený kód a její ovládání je velice intuitivní.
    Základní ovládací prvek aplikace je drawer, který se vysunuje tlačítkem umístěném v pravém horním rohu. V něm jsou obsaženy všechny funkce Gitu. Tedy jeho užívání není při porozumění obecného užívání Gitu nijak náročné. Tato aplikace má integrovaný prohlížeč souborů i jejich editování. Ovšem tento editor není dokonalý. Špatně se v něm posouvá kurzor a navíc nemaže konce řádků. Práce s ním je tedy spíše na obtíž. Naštěstí zde autoři přidali i možnost zvolení vlastního editoru z nainstalovaných aplikací.

    \subsubsection{Pocket Git}
    Dále existuje například aplikace \emph{Pocket Git}. Ta je placená a její kód není veřejně přístupný. Využívá integrovaného správce souborů, ale editor již nechává plně na jiných aplikacích. \emph{Pocket Git} má na první pohled přehlednější uživatelské rozhraní. Jednotlivé funkce gitu rozděluje do různých kategorií a vedle souborů přidává ikonku o jeho stavu. Nicméně \emph{Add} a \emph{Commit} jsou natolik integrované do prohlížeče souborů, že jejich správné použití není vůbec intuitivní. Navíc při práci s tou aplikací často narazíte na nejednoznačná chybová hlášení, která neobsahují bližší popis chyby.

    \subsubsection{Termux}
    Pro vývojáře upřednostňující příkazový řádek je možnost instalace aplikace \emph{Termux} a nainstalování Gitu do prostředí jeho terminálu. Tam je i možné doinstalovat rozšíření \emph{git-lfs} a \emph{git-annex}. \emph{Git-lfs} lde doinstalovat přímo jako balíček. \emph{Git-annex} je možné stáhnout z \url{https://git-annex.branchable.com/} a dle návodu uvést do provozu. Obě tato rozšíření lze ovládat z příkazové řádky, přičemž \emph{git-annex} i přes uživatelské rozhraní. To je implementováno v prohlížeči. Tato webová aplikace je přehledná i pro mobilní zařízení a umožňuje synchronizaci souborů mezi repozitáři různých zařízení.

    \subsection {Desktop}
    \subsubsection{GitKraken}
    Na Linux či Windows existuje mnoho aplikací, které práci s repozitáři zvládají velice dobře. Nicméně prostředí Androidu je od toho desktopového natolik rozdílné, že prostor pro inspiraci je značně omezený. Dobré zkušenosti mám například s aplikací \emph{GitKraken}. Ta zobrazuje repozitář přehledně ve stromové struktuře. V ní lze přímo najetím myši na uzel provádět změny. Funkce Gitu má přehledně zobrazené v horním panelu. Navíc jsou zde dobře řešeny konflikty v souborech. Na jedné straně obrazovky vidíte jednu verzi a na druhé straně druhou. Ve spodní části obrazovky se generuje nová verze. Tu vytváříte postupným procházením obou současných verzí a vybíráním vyhovující varianty. \emph{GitKraken} umí pracovat i s \emph{git-lfs}. K ovládání takto sledovaných souborů používá zvláštní vysouvací nabídku s funkcemi \emph{git-lfs}. Ta se v případě práce s repozitářem podporující toto rozšíření zobrazí vedle základních funkcí. Které soubory takto sleduje lze měnit v nastavení repozitáře či při přidávání souborů do stage.

    \subsubsection{Ungit}
    Na první pohled dobrým dojmem působí i aplikace \emph{Ungit}. Ta vás při každé akci naviguje krok po kroku a usnadňuje tak používání Gitu pro méně zkušené uživatele. Jedná se o webovou aplikaci založenou na \emph{node.js}. Pro její instalaci je třeba příkazová řádka, pro spuštění pak webový prohlížeč. Její hlavní výhoda je tedy nezávislost na platformě. Její ovládání je rychlé, jelikož aplikace zjednodušuje určité procedury Gitu. Například sama nabízí \emph{Commit} bez nutnosti přidávát soubory do \emph{Stage}. Nicméně aplikace tím zapouzdřuje většinu funkcí. Na základní obrazovce kromě stromu změn repozitáře není další ovládací prvek a aplikace se tak v konečném důsledku jeví až příliš uzavřeně.

    \section{Zhodnocení průzkumu}
    Z testování aplikací vyplynulo, že nejjednodušší způsob práce s Gitem je tehdy, když aplikace transparentně zobrazuje funkce Gitu a jejich použití nechá na uživateli. Předejde se tím chybám, jejichž hlášení nejsou vždy dostačující k vyřešení problému. Pokud je funkce dobře zpracována, není třeba vést uživatele krok po kroku. Ovládání se tak urychlí a je stále přehledné.

    Testované aplikace často využívají vlastní textový editor či správce souborů. V obou případech tyto aplikace integrují velice jednoduché verze a jejich použitelnost je tak značně omezená.

    Dalším bodem jsou chybová hlášení. Těm by měla aplikace pokud možno předcházet. Pokud chybě již není vyhnutí, alespoň by měla mít dobrý popis a nebo i návrh jejího řešení.

%------------CHAPTER--------------
\chapter{Návrh}
\chapter{Použité technologie}